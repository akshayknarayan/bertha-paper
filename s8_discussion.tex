\section{Discussion and Conclusion}\label{s:discussion}
\paragrapha{Porting Existing Libraries to \name}
To use \name, application developers must find \tunnels that implement the functionality they require.
At present, this might require them to port existing libraries to \name.
Porting most libraries requires minimal effort (as discussed in \S\ref{s:impl}): the developer calls the appropriate library functions from the \tunnel's implementation.
However, building \tunnels is more complicated in two cases: the first is when trying to encapsulate systems, \eg Shenango~\cite{shenango,caladan} and Shinjuku~\cite{shinjuku}, that provide specialized runtimes rather than composable libraries; the second is when dealing with libraries like OpenSSL~\cite{openssl} which limit how they can be composed (OpenSSL uses sockets internally). 
While we found that we could always encapsulate these runtimes, doing so required taking approaches such as running them in a different thread or process and communicating with them using IPC or memory buffers, adding to \name's overhead.

\paragrapha{Deploying \name} 
Because \name introduces a negotiation protocol, all parties in a connection must use it to be able to communicate.
This negotiation protocol can replace existing ad-hoc methods of feature discovery, such as the QUIC discovery convention described in \S\ref{s:app:quic}.
Of course, to use \name's negotiation, an endpoint must be aware of it; past work has described various mechanisms for advertising protocol availability such as via DNS, and any such mechanism would work with \name.
While we acknowledge that using \name between arbitrary Internet hosts may be challenging, there are common cases where a single entity controls all connection participants, such as in-service proxies,
a cloud provider's SDK, or for internal communication between components of a distributed system.

\paragrapha{Conclusion}
\name contributes abstractions that allow applications to reconfigure their communication functionality at runtime.
While using communication libraries with deployment-specific optimizations without \name requires error-prone out-of-band configuration, \name's \tunnel abstraction enables safe reconfiguration while making it easy to incorporate new functionality and implementations into applications' network use.
