\section{\name Extensions}\label{s:impl}
We implemented \name in Rust. While our implementation uses several Rust features, including traits, async\footnote{
We implement most \tunnel operations, including \texttt{connect\_wrap()}, \texttt{send()}, and \texttt{recv()}, as async functions. We elide this (as well as Rust's lifetime annotations) in code listings for clarity. We use the popular Tokio~\cite{tokio-rs} async runtime for scheduling the resulting coroutines.
}, and the type system, we believe our ideas are more general and can be implemented using similar features and metaprogramming support in other languages. 
For example, we believe C++ templates, Racket macros, and Go's reflection features would all allow us to implement the features we have described in those languages.

We implement \tunnel stacks with a structure similar to Lisp's \texttt{cons}. We represent \texttt{select} simply as a struct that contains both branches. 
Only unilateral \texttt{select}s implement \texttt{connect\_wrap}; \name uses a Rust enum to represent the two choices and returns an instance of this enum.
This enum represents a limited form of dynamic dispatch; at runtime, the connection dispatches to the appropriate implementations of the \texttt{send()} and \texttt{recv()} functions based on the enum's variant.
Multilateral \texttt{select}s cannot directly implement \texttt{connect\_wrap}, since they must use negotiation to pick an implementation branch.
Once the negotiation process returns a choice, the datapath for these multilateral \texttt{select}s is the same as the unilateral case.
We implemented \name's core libraries in \textasciitilde$5,000$ lines of Rust, including the negotiation protocol (of those, \textasciitilde$4,000$ lines).

For the remainder of this section, we describe two optimizations we implemented in \name: zero-rtt negotiation (\S\ref{s:impl:zerortt}), and lock-free single-host reconfiguration (\S\ref{s:impl:fast-reconfig}).

\subsection{Zero-RTT Negotiation}\label{s:impl:zerortt}
As we note in \S\ref{s:negotiation}, runtime-reconfigurability requires agreement, and hence endpoints must communicate before they can transfer data. The point-to-point protocol described in \S\ref{s:negotiation} completes in one RTT. 
To address this, the \name implements an additional optimization that allows clients to reestablish connections without additional negotiation, thus reducing overheads for applications that use many short-lived connections. Our approach to doing so is inspired by QUIC's zero-RTT~\cite{quic} connection establishment.

Zero-RTT negotiation requires the client to remember the datapath stack used by previous connections and re-use it when reconnecting. 
The client sends the server a zero-RTT negotiation message when it knows of a previously negotiated stack and then immediately instantiates that stack; the client can then send data. 
When the server receives a zero-RTT negotiation message, it checks if the previously negotiated stack can still be used. If so, the server re-initiates the stack and uses it to process all subsequent data (including any in-flight data from the client). If, however, the server cannot use the stack, it sends the client a message indicating that the negotiation has failed and proposes a new stack. If \name on the client receives such a message indicating negotiation failure, it tears down the existing stack and instantiates the stack included in the failure message instead.

\subsection{Faster Reconfiguration}\label{s:impl:fast-reconfig}

Changing one or more \tunnel implementations during reconfiguration requires coordinating with any thread that might be using the connection. This coordination is necessary for safety.
The approach we described in \S\ref{s:impl-reconfig} used locks to implement this coordination. While using locks ensures safety, as we show in \S\ref{s:eval:reconfig}, this adds overheads in the common case when no reconfiguration occurs.

We thus also implement a different approach with lower fast-path overheads: during reconfiguration, we wait for all application threads to arrive at a barrier. Once the barrier resolves, threads other than the one performing the reconfiguration wait on a channel before performing any further operations on the encapsulated connection. Thus, we know it is safe to modify the encapsulated connection by simply writing to the appropriate memory\footnote{We use Rust's \texttt{UnsafeCell} to do this.}, since we know there will be no concurrent accesses to the connection object. 
The switching point, in this case, is the barrier's resolution: after this, no connection will use the old connection objects.
Note that in the case of multilateral \tunnels, the barrier must protect the negotiation phase because the negotiation protocol uses the connection to communicate.
This approach is similar to that used by stop-the-world garbage collectors. The only overhead on the fast path is that when sending or receiving data, each thread needs to read a boolean value indicating whether it needs to stop at a barrier.
We evaluate reconfiguration mechanisms in \S\ref{s:eval:reconfig}.

    
\eat{
\begin{outline}
\0 Changing one or more \tunnel implementations during reconfiguration requires coordinating with any thread that might be using the connection. This coordination is necessary for safety.
    \1 In \S\ref{s:impl-reconfig}, our approach used locks for this. While the use of locks ensures safety, as we show in \fixme{XXX} this adds overheads in the application's datapath.
\0 Our implementation uses a different approach with lower fast-path overheads: during reconfiguration, we wait for all application threads to arrive at a barrier before switching to a new implementation.
    \1 This approach is similar to that used by stop-the-world garbage collectors. The only overhead on the fast path is that when sending or receiving data, each thread needs to read a boolean indicating that it needs to stop at a barrier.
    \1 However, this optimization increases reconfiguration latency, that is time until the new implementation is used.
\0 \fixme{We must use a similar barrier that waits for all endpoints to switch (e.g., in multiparty negotiation) or risk safety?}
\end{outline}
}

\cut{
\subsection{\tunnel Stack Optimizations}\label{s:impl:optimization}
The amount of computation performed when sending or receiving data over a connection depends on its \tunnel stack, and the implementations chosen.
In some cases, implementation choices might make redundant some \tunnels in the connection's \tunnel stack, and in other cases reordering \tunnels might result in better performance.
Unfortunately, the specific optimizations that should be applied depend on the connection and on the chosen implementations, which makes it impractical for developers to manually apply these optimizations.
Further, since \name is extensible and cannot know the set of \tunnels it will be working with ahead of time, it cannot implement these optimizations automatically.

\name instead takes an approach inspired by Broadway~\cite{broadway} 
where \tunnel implementors and application developers can provide \emph{optimization passes}.
These passes work on subsets of the \tunnel stack, and can modify them. These passes are similar to optimization passes in machine learning frameworks~\cite{tensorflow-xla, onnx, tvm}, data analytics~\cite{weld}, and Click~\cite{click-opt}.
Passes are used at compile time: \name enumerates all possible \tunnel stacks that can be used by a connection and applies optimizations to them. This is because the passes themselves rely on type information (to identify \tunnel and \tunnel implementations), and this information is not available at runtime.
We believe we can implement runtime optimizations instead by recording type information (similar to RTTI), but we leave this to future work.

Note that \name assumes that optimization passes are correct.
While the type system does check that substitutions are compatible (\eg disallowing substitutions that change the input or output datatype), it can neither ensure that semantics remain the same nor reason about performance.
Note that negotiation will still check the post-optimization \tunnel stack for multilateral compatibility.
Therefore, we require application developers to analyze these effects, and then explicitly opt into optimization passes.
Thus, the effect here is to reduce the tedium and effort required to optimize an application. 
}
    
\eat{
\begin{outline}
\0 The amount of computation performed when sending or receiving data over a connection depends on its \tunnel stack, and the implementations chosen.
    \1 In some cases, implementation choices might make redundant some \tunnels in the connection's \tunnel stack, and in other case reordering \tunnels might yield better performance.
    \1 Unfortunately, the specific optimizations that should be applied depend on the connection and on the chosen implementations.
        \2 Makes it impractical to have developers manually apply these optimizations.
        \2 Also cannot be implemented by \name, which is extensible and allows \tunnels and \tunnel implementations to be added later.
\0 Take an approach, inspired by Broadway~\cite{guyerlin} %
  where \tunnel implementors and application developers can provide optimization passes.
    \1 These passes work on subsets of the \tunnel stack, and can modify them. These passes look similar to optimization passes in machine learning frameworks~\cite{tensorflow-xla, onnx, tvm}, data analytics~\cite{weld}, and Click~\cite{click-opt}.
    \1 Passes are used at compile time: \name enumerates all possible \tunnel stacks that can be used by a connection, and applies optimizations to them. This is because the passes themselves rely on type information (to identify \tunnel and \tunnel implementations), and this information is not available at runtime.
    \1 We believe we can implement runtime optimizations instead by recording type information (similar to RTTI), but we leave this to future work.
\0 Note that \name assumes that optimization passes are correct.
    \1 While the type system does check that substitutions are compatible (\eg disallowing substitutions that change the input or output datatype), it can neither ensure that semantics remain the same nor reason about performance.
      \2 Note that negotiation will still check the post-optimization \tunnel stack for multilateral compatibility.
    \1 Therefore, we require application developers to analyze these effects, and then explicitly opt into optimization passes.%
    \1 Thus, the effect here is to reduce the tedium and effort required to optimize an application. Our evaluation in \S\ref{s:eval:kv} shows an example (Figure~\ref{f:kvlb}) where we use optimization passes to improve application performance.
\end{outline}
}
